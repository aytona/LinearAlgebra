\documentclass[]{article}
\usepackage{amsmath}

\hbadness=99999
\title{Linear Algebra: Week 1 Notes and Exercises}
\author{Christopher Aytona}

\begin{document}

\begin{center}
\Huge{Week 1 Notes and Exercises}
\end{center}

\section{Notes}
\textbf{Network Flow Diagrams}\\
$400 + x_2 = x_1$\\
$400 = x_1 - x_2$\\
$x_1 + x_3 - x_4 = 600$\\
$x_4 + x_5 = 100$\\
$x_2 + x_3 + x_5 = 300$\\

\textbf{Solutions to Systems of Linear Equations}\\
\textsf{A unique solution(Consistent)}\\
$x - 2y = -1$\\
$x = 2y-1$\\
$-2y+1+3y = 3$\\
$y = 2$\\
$x - 4 = -1$\\
$x = 3$\\
$(x, y) = (3, 2)$\\
\textsf{Infinitely many solutions(Consistent)}\\
$x - 2y = -1$\\
$-x+2y=3$\\
$x = 2y-1$\\
$-2y+1+2y=3$\\
$0y=2$\\
\textsf{No solutions(Inconsistent)}\\
$x-2y=-1$\\
$-x+2y=1$\\
$x=2y-1$\\
$-2y+1+2y=1$\\
$0y=0$\\

\textbf{Linear Equations}\\
\textsf{Method 1}\\
1) $x_1 - 2x_2 = -1$\\
2) $-x_1 + 3x_2 = 3$\\
Rewrite (1) as:\\
3) $x_1 = -1 + 2x_2$\\
Sub (3) into (2):\\
$-(-1+2x_2) + 3x_2 = 3$\\
$1 - 2x_2 + 3x_2 +2 = 3$\\
$1 + x_2 = 3$\\
$x_2 = 3 - 1$\\
4) $x_2 = 2$\\
Sub (4) into (1):\\
$x_1 - 2(2) = -1$\\
$x_1 - 4 = -1$\\
$x_1 = -1 + 4$\\
$x_1 = 3$\\
\textsf{Method 2}\\
1) $x_1 - 2x_2 + x_3 = 0$\\
2) $2x_2 - 8x_3 = 8$\\
3) $-4x_2 + 5x_2 + x_3 = -9$\\
Multiply (1) by 4\\
$4x_1 - 8x_2 + 4x_3 = 0$\\
$-4x_1 + 5x_2 + x_3 = -9$\\
Add both\\
$-3x_2 + 5x_3 = -9$\\
$2x_2 - 8x_3 = 8$\\
Make the coefficient the same(By multiplication)\\
$-6x_2 + 10x_3 = -18$\\
$6x_2 - 24x_3 = 24$\\
Addition\\
$-14x_3 = 6$\\
$x_3 = \frac{-6}{14}$\\

\textbf{Exercise}\\
\textsf{Problem 1}\\
1) $x_1 - 3x_3 = 8$\\
2) $2x_1 + 2x_2 + 9x_3 = 7$\\
3) $x_2 + 5x_3 = -2$\\
$x_1 = 8 + 3x_3$\\
$x_2 = -2 - 5x_3$\\
Sub\\
$2(8 + 3x_3) + 2(-2 - 5x_3) + 9x_3 = 7$\\
$16 + 6x_3 -4 -10x_3 + 9x_3 = 7$\\
$6x_3 - 10x_3 + 9x_3 = 7 - 16 + 4$\\
$5x_3 = -5$\\
$x_3 = -1$\\
$x_1 - 3(-1) = 8$\\
$x_1 = 8 - 3$\\
$x_1 = 5$\\
$2x_2 + 18(-1) = -9$\\
$2x_2 = 6$\\
$x_2 = 3$\\
$(x_1, x_2, x_3) = (5, 3, -1)$\\
\textsf{Problem 2}\\
1) $x_2 + 4x_3 = -5$\\
2) $x_1 + 3x_2 + 5x_3 = -2$\\
3) $3x_1 + 7x_2 + 7x_3 = 4$\\
$x_2 = -5 - 4x_3$\\
$x_1 = -3(-5 - 4x_3) - 5x_3$\\
$x_1 = 15 + 4x_3 - 5x_3$\\
$x_1 = -x_3 + 15$\\
$3(-x_3 + 15) + 7(-5 - 4x_3) + 7x_3 = 4$\\
$-3x_3 + 45 - 35 - 28x_3 + 7x_3 = 4$\\
$-24x_3 = -6$\\
$x_3 = \frac{1}{4}$\\
$x_2 + 4(\frac{1}{4}) = -5$\\
$x_2 = -6$\\
$x_1 + 3(-6) + 5(\frac{1}{4}) = -2$\\
$x_1 = -2 + 18 - \frac{5}{4}$\\
$x_1 = 15\frac{1}{4}$\\
$(x_1, x_2, x_3) = (15\frac{1}{4}, -6, \frac{1}{4})$\\

\textbf{Row Operations}\\
There are 3 types of row operations that can be performed on an augmented matrix without changing the solutions.\\
$\cdot$ Interchanging two rows.\\
$\cdot$ Scaling a row.\\
$\cdot$ Adding/Subtracting a row to another.\\

\textbf{Reduced Row Echelon Form}\\
$\cdot$ All non-zero rows are above any rows of all zeros.\\
$\cdot$ The leading non-zero coefficient of a non-zero row is a 1 and is strictly to the right of the leading 1 in the row above it.\\
$\cdot$ Each leading 1 is the only non-zero entry in that column.\\

\textbf{Matrices Example}\\
\[
\begin{bmatrix}
	1 & -2 & 1 & | & 0 \\
	0 & 2 & -8 & | & 8 \\
	-4 & 5 & 9 & | & -9
\end{bmatrix}
\]\\
$4R_1 + R_3$\\
\[
\begin{bmatrix}
	1 & -2 & 1 & | & 0 \\
	0 & 2 & -8 & | & 8 \\
	0 & -3 & 13 & | & -9
\end{bmatrix}
\]\\
$\frac{1}{2}R_2$\\
\[
\begin{bmatrix}
	1 & -2 & 1 & | & 0 \\
	0 & 1 & -4 & | & 4 \\
	0 & -3 & 13 & | & -9
\end{bmatrix}
\]\\
$3R_2 + R_3$\\
\[
\begin{bmatrix}
	1 & -2 & 1 & | & 0 \\
	0 & 1 & -4 & | & 4 \\
	0 & 0 & 1 & | & 3 \\
\end{bmatrix}
\]
$R_2 + 4R_3$\\
\[
\begin{bmatrix}
	1 & -2 & 1 & | & 0 \\
	0 & 1 & 0 & | & 16 \\
	0 & 0 & 1 & | & 3
\end{bmatrix}
\]\\
$R_1 - R_3$\\
\[
\begin{bmatrix}
	1 & -2 & 0 & | & -3 \\
	0 & 1 & 0 & | & 16 \\
	0 & 0 & 1 & | & 3
\end{bmatrix}
\]\\
\[
\begin{bmatrix}
	1 & 0 & 0 & | & 29 \\
	0 & 1 & 0 & | & 16 \\
	0 & 0 & 1 & | & 3
\end{bmatrix}
\]\\
$x_1 = 29 , x_2 = 16 , x_3 = 3$\\
\textsl{More examples}\\
\[
\begin{bmatrix}
	1 & 0 & 1 & | & 32 \\
	0 & 1 & 2 & | & 16 \\
	0 & 0 & 0 & | & 0
\end{bmatrix}
\]\\
It is a reduced row echelon.\\
Let $x_3 = S$\\
$x_1 = 32 - S$\\
$x_2 = 16 - 2S$\\
$x_3 = 0 + S$\\

\section{Exercise 1.1}
Solve each system in Exercise 1-4 by using elementary row operations on the equations or on the augmented matrix. Follow the systematic elimination procedure described in this section.\\

1) $x_1 + 5x_2 = 7$, $-2x_1 - 7x_2 = -5$\\
$x_1 = -5x_2 + 7$\\
$-2(-5x_2 + 7) - 7x_2 = -5$\\
$10x_2 - 14 - 7x_2 = -5$\\
$3x_2 = 9$\\
$x_2 = 3$\\
$x_1 = -5(3) + 7$\\
$x_1 = -15 + 7$\\
$x_1 = -8$\\
\[
\begin{bmatrix}
	1 & 5 & | & 7 \\
	-2 & -7& | & -5
\end{bmatrix}
\]\\
$R_2 + 2R_1$\\
\[
\begin{bmatrix}
	1 & 5 & | & 7 \\
	-2+2(1) & -7+2(5) & | & -5+2(7)
\end{bmatrix}=
\begin{bmatrix}
	1 & 5 & | & 7 \\
	0 & 3 & | & 9
\end{bmatrix}
\]\\
$\frac{1}{3}R_2$\\
\[
\begin{bmatrix}
	1 & 5 & | & 7 \\
	\frac{1}{3}(0) & \frac{1}{3(3)} & | & \frac{1}{3}(9)
\end{bmatrix}=
\begin{bmatrix}
	1 & 5 & | & 7 \\
	0 & 1 & | & 3
\end{bmatrix}
\]\\
$R_1 + -5R_2$\\
\[
\begin{bmatrix}
	1 + -5(0) & 5 + -5(1) & | & 7 + -5(3) \\
	0 & 1 & | & 3
\end{bmatrix}=
\begin{bmatrix}
	1 & 0 & | & -8 \\
	0 & 1 & | & 3
\end{bmatrix}
\]\\
$$(x_1,x_2) = (-8,3)$$\\

2) $3x_1 + 6x_2 = -3$, $5x_1 + 7x_2 = 10$\\
$3x_1 = -6x_2 - 3$\\
$x_1 = -2x_2 - 1$\\
$5(-2x_2 - 1) + 7x_2 = 10$\\
$-10x_2 - 5 + 7x_2 = 10$\\
$-3x_2 = 15$\\
$x_2 = -5$\\
$x_1 = -2(-5) - 1$\\
$x_1 = 9$\\
\[
\begin{bmatrix}
	3 & 6 & | & -3 \\
	5 & 7 & | & 10
\end{bmatrix}
\]\\
$\frac{1}{3}R_1$
\[
\begin{bmatrix}
	\frac{1}{3}(1) & \frac{1}{3}(2) &|& \frac{1}{3}(-1)\\
	5&7&|&10
\end{bmatrix}=
\begin{bmatrix}
	1&2&|&-1\\
	5&7&|&10
\end{bmatrix}
\]\\
$R_2-5R_1$
\[
\begin{bmatrix}
	1&2&|&-1\\
	5-5(1)&7-5(2)&|&10-5(-1)
\end{bmatrix}=
\begin{bmatrix}
	1&2&|&-1\\
	0&-3&|&15
\end{bmatrix}
\]\\
$-\frac{1}{3}R_2$
\[
\begin{bmatrix}
	1&2&|&-1\\
	-\frac{1}{3}(0)&-\frac{1}{3}(-3)&|&-\frac{1}{3}(15)
\end{bmatrix}=
\begin{bmatrix}
	1&2&|&-1\\
	0&1&|&-5
\end{bmatrix}
\]\\
$R_1-2R_2$
\[
\begin{bmatrix}
	1-2(0)&2-2(1)&|&-1-2(-5)\\
	0&1&|&-5
\end{bmatrix}=
\begin{bmatrix}
	1&0&|&9\\
	0&1&|&-5
\end{bmatrix}
\]\\
$$(x_1, x_2) = (9,-5)$$\\

3) Find the point $(x_1, x_2)$ that lies on the line $x_1+2x_2=4$ and on the line $x_1-x_2=1$.\\
$x_1 + 2x_2 = 4$\\
$x_1-x_2=1$\\
\[
\begin{bmatrix}
	1 & 2 & | & 4 \\
	1 & -1 & | & 1
\end{bmatrix}
\]\\
$R_2 - R_1$\\
\[
\begin{bmatrix}
	1 & 2 & | & 4 \\
	1-1 & -1-2 & | & 1 - 4
\end{bmatrix}=
\begin{bmatrix}
	1 & 2 & | & 4\\
	0 & -3 & | & -3
\end{bmatrix}
\]\\
$-\frac{1}{3}R_2$\\
\[
\begin{bmatrix}
	1 & 2 & | & 4\\
	-\frac{1}{3}(0) & -\frac{1}{3}(-3) & | & -\frac{1}{3}(-3)
\end{bmatrix}=
\begin{bmatrix}
	1 & 2 & | & 4 \\
	0 & 1 & | & 1
\end{bmatrix}
\]\\
$R_1 - 2R_2$\\
\[
\begin{bmatrix}
	1 - 2(0) & 2 - 2(1) & | & 4 - 2(1)\\
	0 & 1 & | & 1
\end{bmatrix}=
\begin{bmatrix}
	1 & 0 & | & 2\\
	0 & 1 & | & 1
\end{bmatrix}
\]\\
$$(x_1, x_2) = (2, 1)$$\\

4) Find the point of intersection of the lines $x_1+2x_2=-13$ and $3x_1-2x_2=1$.\\
$x_1+2x_2=-13$\\
$3x_1-2x_2 = 1$\\
\[
\begin{bmatrix}
	1&2&|&-13\\
	3&-2&|&1
\end{bmatrix}
\]\\
$R_2-3R_1$
\[
\begin{bmatrix}
	1&2&|&-13\\
	3-3(1)&-2-3(2)&|&1-3(-13)
\end{bmatrix}=
\begin{bmatrix}
	1&2&|&-13\\
	0&-8&|&40
\end{bmatrix}
\]\\
$-\frac{1}{8}R_2$
\[
\begin{bmatrix}
	1&2&|&-13\\
	-\frac{1}{8}(0)&-\frac{1}{8}(-8)&|&-\frac{1}{8}(40)
\end{bmatrix}=
\begin{bmatrix}
	1&2&|&-13\\
	0&1&|&-5
\end{bmatrix}
\]\\
$R_1-2R_2$
\[
\begin{bmatrix}
	1-2(0)&2-2(1)&|&-13-2(-5)\\
	0&1&|&-5
\end{bmatrix}=
\begin{bmatrix}
	1&0&|&-3\\
	0&1&|&-5
\end{bmatrix}
\]\\
$$(x_1, x_2) = (-3, -5)$$\\

17) Do the three lines $2x_1+3x_2=-1$, $6x_1+5x_2=0$, and $2x_1-5x_2=7$ have a common point of intersection? Explain.\\
\[
\begin{bmatrix}
	2&3&|&-1\\
	6&5&|&0\\
	2&-5&|&7
\end{bmatrix}
\]
$R_2-3R_1$ and $R_3-R_1$
\[
\begin{bmatrix}
	2&3&|&-1\\
	6-3(2)&5-3(3)&|&0-3(-1)\\
	2-2&-5-3&|&7-(-1)
\end{bmatrix}=
\begin{bmatrix}
	2&3&|&-3\\
	0&-4&|&3\\
	0&-8&|&8
\end{bmatrix}
\]\\
$R_3-2R_2$
\[
\begin{bmatrix}
	2&3&|&-3\\
	0&-4&|&3\\
	0-2(0)&-8-2(-4)&|&8-2(3)
\end{bmatrix}=
\begin{bmatrix}
	2&3&|&-3\\
	0&-4&|&3\\
	0&0&|&2
\end{bmatrix}
\]\\
Therefore, the three lines doesn't have a common point of intersection because the third equation shows the system is inconsistent with $0=2$\\

18) Do the three planes $2x_1+4x_2+4x_3=4$, $x_2-2x_3=-2$, and $2x_1+3x_2=0$ have at least one common point of intersection? Explain.\\
\[
\begin{bmatrix}
	2&4&4&|&4\\
	0&1&-2&|&-2\\
	2&3&0&|&0
\end{bmatrix}	
\]\\
$R_3-R_1$
\[
\begin{bmatrix}
	2&4&4&|&4\\
	0&1&-2&|&-2\\
	2-2&3-4&0-4&|&0-4
\end{bmatrix}=
\begin{bmatrix}
	2&4&4&|&4\\
	0&1&-2&|&-2\\
	0&-1&-4&|&-4
\end{bmatrix}
\]\\
$R_3+R_2$
\[
\begin{bmatrix}
	2&4&4&|&4\\
	0&1&-2&|&-2\\
	0+0&-1+1&-4+(-2)&|&-4+(-2)
\end{bmatrix}=
\begin{bmatrix}
	2&4&4&|&4\\
	0&1&-2&|&-2\\
	0&0&-6&|&-6
\end{bmatrix}
\]\\
$\frac{1}{2}R_1$ and $-\frac{1}{6}R_3$
\[
\begin{bmatrix}
	\frac{1}{2}(2)&\frac{1}{2}(4)&\frac{1}{2}(4)&|&\frac{1}{2}(4)\\
	0&1&-2&|&-2\\
	-\frac{1}{6}(0)&-\frac{1}{6}(0)&-\frac{1}{6}(-6)&|&-\frac{1}{6}(-6)
\end{bmatrix}=
\begin{bmatrix}
	1&2&2&|&2\\
	0&1&-2&|&-2\\
	0&0&1&|&1
\end{bmatrix}
\]\\
Therefore, since the Reduced Row Echelon form of a matrix has a leading 1 in each row, then the corresponding system is consistent and has at least one solution.\\

In Exercise 19-22, determine the value(s) of $h$ such that the matrix is the augmented matrix of a consistent linear system.\\

19) \[
\begin{bmatrix}
	1 & h &|& 4\\
	3 & 6 &|& 8
\end{bmatrix}
\]
$R_2-3R_1$
\[
\begin{bmatrix}
	1&h&|&4\\
	3-3(1)&6-3(h)&|&8-3(4)
\end{bmatrix}=
\begin{bmatrix}
	1&h&4\\
	0&3h-6&|&-4
\end{bmatrix}
\]\\
If $h=2$, then the system has no solution, because $3(2)-6=0$ cannot equal $-4$. Otherwise, if $h\neq2$, the system has a solution.\\

20) \[
\begin{bmatrix}
	1&h&|&-5\\
	2&-8&|&6
\end{bmatrix}
\]
$R_2-2R_1$
\[
\begin{bmatrix}
	1&h&|&-5\\
	2-2(1)&-8-2(h)&|&6-2(-5)
\end{bmatrix}=
\begin{bmatrix}
	1&h&|&-5\\
	0&-2h-8&|&16
\end{bmatrix}
\]\\
If $h=-4$, then the system has no solution, because $-2(-4)-8=0$ cannot equal $16$. Otherwise, if $h\neq-4$, the system has a solution.\\

21) \[
\begin{bmatrix}
	1&4&|&-2\\
	3&h&|&-6
\end{bmatrix}
\]
$R_2-3R_1$
\[
\begin{bmatrix}
	1&4&|&-2\\
	3-3(1)&h-3(4)&|&-6-3(-2)
\end{bmatrix}=
\begin{bmatrix}
	1&4&|&-2\\
	0&h-12&|&0
\end{bmatrix}
\]\\
The system has infinite solutions because $R_2 = 0$.\\

22) \[
\begin{bmatrix}
	-4&12&|&h\\
	2&-6&|&-3
\end{bmatrix}
\]
$R_2+\frac{1}{2}R_1$
\[
\begin{bmatrix}
	-4&12&|&h\\
	2+\frac{1}{2}(-4)&-6+\frac{1}{2}(12)&|&-3+\frac{h}{2}
\end{bmatrix}=
\begin{bmatrix}
	-4&12&|&h\\
	0&0&|&-3+\frac{h}{2}
\end{bmatrix}
\]\\
The system is only consist if and only if $h=6$.\\

23) \\

a) Every elementary row operation is reversible.\\
True, scaling and adding or subtracting rows are reversible.\\

b) A $5\times6$ matrix has six rows.\\
False, a $5\times6$ matrix has five rows and six columns.\\

c) The solution set of a linear system involving variables $x_1, \cdots, x_n$ is a list of numbers $(s_1,\cdots,s_n)$ that makes each equation in the system a true statement when the values $s_1,\cdots,s_n$ are substituted for $x_1,\cdots, x_n$ respectively.\\
False, the solution set consists of all possible solutions as a statement is only true if it is always true. \\

d) Two fundamental questions about a linear system involve existence and uniqueness.\\
True, uniqueness implies existence and existence implies uniqueness. Then $A$ is invertible.\\

24) \\

a) Two matrices are row equivalent if they have the same number of rows.\\
False, row equivalent requires a sequence of row operations that transforms one matrix into another.\\

b) Elementary row operations on an augmented matrix never change the solution set of the associated linear system.\\
True.\\

c) Two equivalent linear systems can have different solutions sets.\\
False.\\

d) A consistent system of linear equations has one or more solutions.\\
True, a consistent system has at least one solution.\\

25) Find an equation involving $g, h,$ and $k$ that makes this augmented matrix correspond to a consistent system:
\[
\begin{bmatrix}
	1&-4&7&|&g\\
	0&3&-5&|&h\\
	-2&5&-9&|&k
\end{bmatrix}
\]\\
$R_3+2R_1$
\[
\begin{bmatrix}
	1&-4&7&|&g\\
	0&3&-5&|&h\\
	-2+2(1)&5+2(-4)&-9+2(7)&|&k+2g
\end{bmatrix}=
\begin{bmatrix}
	1&-4&7&|&g\\
	0&3&-5&|&h\\
	0&-3&5&|&k+2g
\end{bmatrix}
\]\\
$R_3+R_2$
\[
\begin{bmatrix}
	1&-4&7&|&g\\
	0&3&-5&|&h\\
	0&-3+3&5+(-5)&|&k+2g+h
\end{bmatrix}=
\begin{bmatrix}
	1&-4&7&|&g\\
	0&3&-5&|&h\\
	0&0&0&|&k+2g+h
\end{bmatrix}
\]\\
Therefore, this system is consistent if and only if $k+2g+h = 0$.\\

26) Suppose the system below is consistent for all possible values of $f$ and $g$. What can you say about the coefficients $c$ and $d$? Justify your answer.\\
$2x_1+4x_2=f$\\
$cx_1+dx_2=g$\\
\[
\begin{bmatrix}
	2&4&|&f\\
	c&d&|&g
\end{bmatrix}
\]\\
$\frac{1}{2}R_1$
\[
\begin{bmatrix}
	1&2&|&\frac{f}{2}\\
	c&d&|&g
\end{bmatrix}
\]\\
$R_2 = R_2-cR_1$
\[
\begin{bmatrix}
	1&2&|&\frac{f}{2}\\
	c-c(1)&d-c(2)&|&g-c\frac{f}{2}
\end{bmatrix}=
\begin{bmatrix}
	1&2&|&\frac{f}{2}\\
	0&d-2c&|&g-c\frac{f}{2}
\end{bmatrix}
\]\\

In Exercise 29-32, find the elementary row operation that transforms the first matrix into the second, and then find the reverse row operation that transforms the second matrix into the first.\\

29) $R_1$ swap with $R_3$.\\
\[
\begin{bmatrix}
0&-2&|&5\\
1&3&|&-5\\
3&-1&|&6
\end{bmatrix}
\begin{bmatrix}
3&-1&|&6\\
1&3&|&-5\\
0&-2&|&5
\end{bmatrix}
\]\\


30)$-\frac{1}{5}R_3$.\\
\[
\begin{bmatrix}
1&3&|&-4\\
0&-2&|&6\\
0&-5&|&10
\end{bmatrix}
\begin{bmatrix}
1&3&|&-4\\
0&-2&|&6\\
0&1&|&-2
\end{bmatrix}
\]\\

\section{Exercise 1.2}

In Exercise 1 and 2, determine which matrices are in reduced echelon form and which others are only in echelon form.\\

1) a) \[
\begin{bmatrix}
1&0&0&0\\
0&1&0&0\\
0&0&1&1
\end{bmatrix}
\]
Reduced echelon form because each leading 1 is the only non-zero entry in that column.\\

b) \[
\begin{bmatrix}
1&0&1&0\\
0&1&1&0\\
0&0&0&1
\end{bmatrix}
\]
Reduced echelon form because the leading non-zero coefficient of a non-zero row is a 1 and is strictly to the right of the leading 1 in the row above it.\\

c) \[
\begin{bmatrix}
1&0&0&0\\
0&1&1&0\\
0&0&0&0\\
0&0&0&1
\end{bmatrix}
\]
Not echelon form because not all non-zero rows are above any rows of all zeros.\\

d) \[
\begin{bmatrix}
1&1&0&1&1\\
0&2&0&2&2\\
0&0&0&3&3\\
0&0&0&0&4
\end{bmatrix}
\]
Echelon form but not reduced yet.\\
\end{document}