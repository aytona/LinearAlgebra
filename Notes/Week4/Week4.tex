\documentclass[a4paper]{article}

\usepackage{amsmath}
\usepackage{amssymb}

\hbadness=99999
\voffset=-1in
\oddsidemargin=5pt
\textwidth=450pt
\textheight=700pt

\begin{document}
\title{Linear Algebra: Week 4 Notes and Exercises}
\author{Christopher Aytona}
\maketitle

\section{Notes}
Notes\\

Vector Space\\
1) $\vec{0} \in V$\\
2) $\vec{u}, \vec{v} \in V$ then must have $\vec{u} + \vec{v} \in V$\\
3) $c \in \mathbb{R}, \vec{u} \in V$ then must have $c \vec{u} \in V$\\
Exercise\\

2) a) True because scaling $\vec{u}$ with either a positive or negative value will still be in $W$.\\
$\vec{u} = \begin{bmatrix}
x_1\\
y_1
\end{bmatrix}, x_1 \geq 0, y_1 \geq 0$\\
$c \begin{bmatrix}
x_1\\
y_1
\end{bmatrix} = \begin{bmatrix}
cx_1\\
cy_1
\end{bmatrix}$\\
$c \geq 0 \Rightarrow cx_1 \geq 0 \Rightarrow cy_1 \geq 0$\\
$c \leq 0 \Rightarrow cx_1 \leq 0 \Rightarrow cy_1 \leq 0$\\
b)Let $\vec{u} = \begin{bmatrix}
4\\
2
\end{bmatrix}$, $\vec{v} = \begin{bmatrix}
-1\\
-8
\end{bmatrix}$ $\therefore \vec{u} + \vec{v} = \begin{bmatrix}
3\\
-6
\end{bmatrix}$ which is not in any of the quadrants associated with $W \therefore W$ is not a vector space .\\

6) $p(t) = a + t^2$\\
$V = \begin{bmatrix}
x_1&x_2&x_3
\end{bmatrix} \in \mathbb{R}^3 | x_1 \in \mathbb{R}, x_2 = 0, x_3 = 1$\\
$\begin{bmatrix}
a&0&1
\end{bmatrix}$
Its not a vector space because scaling or adding vectors will change the $x_3$.\\

\section{Exercises}

1. Let $V$ be the first quadrant in the $xy$-plane; that is let $V = \{
\begin{bmatrix}
x\\
y
\end{bmatrix} : x \geq 0, y \geq 0 \} $\\
a) If $u$ and $v$ are in $V$, is $u+v$ in $V$? Why?\\
If $\vec{u}, \vec{v} \in V$ is $\vec{u} + \vec{v} \in V$\\
$\vec{u} \in V \Rightarrow \vec{u} = \begin{bmatrix}
x_1\\
y_1
\end{bmatrix}, x_1 \geq 0, y_1 \geq 0$\\
$\vec{v} \in V \Rightarrow \vec{v} = \begin{bmatrix}
x_2\\
y_2
\end{bmatrix}, x_2 \geq 0, y_2 \geq 0$\\
$\vec{u} + \vec{v} = \begin{bmatrix}
x_1\\
y_1
\end{bmatrix} + \begin{bmatrix}
x_2\\
y_2
\end{bmatrix} = \begin{bmatrix}
x_1 + x_2\\
y_1 + y_2
\end{bmatrix} = x_1 + x_2 \geq 0, y_1 + y_2 \geq 0$\\
$\Rightarrow \vec{u} + \vec{v} \in V$\\
b) Find a specific vector $u$ in $V$ and a specific scalar $c$ such that $cu$ is not in $V$. (This is enough to show that $V$ is not a vector space.)\\
Let $\vec{u} = \begin{bmatrix}
1\\
1
\end{bmatrix}$ then $\vec{u} \in V$,\\
Let $c = -1$ then $c\vec{u} = -1\begin{bmatrix}
1\\
1
\end{bmatrix} = \begin{bmatrix}
-1\\
-1
\end{bmatrix} \notin V$\\
2. Let $W$ be the union of the first and third quadrants in the $xy$-plane. That is, let $W = \{\begin{bmatrix}
x\\
y
\end{bmatrix} : xy \geq 0 \}$\\
a) If $u$ is in $W$ and $c$ is any scalar, is $cu$ in W? Why?\\
If $\vec{u} \in W, c \in \mathbb{R}$, is $c\vec{u} \in W$\\
Let $\vec{u} = \begin{bmatrix}
x_1\\
y_1
\end{bmatrix}, x_1y_1 \geq 0$, Let $c \in \mathbb{R}$\\
$cu = c\begin{bmatrix}
x_1\\
y_1
\end{bmatrix} = \begin{bmatrix}
cx_1\\
cy_1
\end{bmatrix}$\\
$(cx_1)(cy_1) = c^2x_1y_1$\\
But $c^2 \geq 0, \forall c \in \mathbb{R}$\\
$c^2x_1y_1 \geq 0 \Rightarrow c\vec{u}\in W$\\
b) Find specific vectors $u$ and $v$ in $W$ such that $u + v$ is not in $W$. This is enough to show that $W$ is not a vector space.\\
Let $\vec{u} = \begin{bmatrix}
1\\
2
\end{bmatrix}$, let $\vec{v} = \begin{bmatrix}
-2\\
-1
\end{bmatrix}$\\
Then $\vec{u} + \vec{v} = \begin{bmatrix}
1-2\\
2-1
\end{bmatrix} = \begin{bmatrix}
-1\\
1
\end{bmatrix} \notin W, since (-1)(1) = -1$\\

In Exercise 5-8, determine if the given set is a subspace of $\mathbb{P}_n$ for an appropriate value of $n$. Justify your answers.\\
All polynomials $p(t) = at^2$, $a \in \mathbb{R}$\\
5. All polynomials of the form $p(t) = at^2$, where $a$ is in $\mathbb{R}$\\
If $a = 0, p(t) = 0t^2 = \vec{0}$ $\therefore \vec{0} \in \{p(t)\}$\\
If $u(t) = a_1t^2$, $v(t) = a_2t^2$, then $\vec{u} + \vec{v} = a_1t^2+a_2t^2 = (a_1+a_2)t^2 \in \{ p(t) \}$\\
If $u(t) = a_1t^2$ and $c \in \mathbb{R}$, $c\vec{u} = ca_1t^2 = (ca_1)t^2 \in \{p(t)\}$\\
$\therefore \{p(t)\}$ is a subspace\\
6. All polynomials of the form $p(t) = a + t^2$, where $a$ is in $\mathbb{R}$\\
$\vec{0} = 0+0t+t^2$ is not in $\{p(t)\}$\\
If $n(t) = a_1 + t^2$, $c = 0$\\
$c\vec{u} = 0(a_1 + t^2) = 0 \notin \{p(t)\}$\\
$\therefore$ not in subspace\\
7. All polynomials of degree at most 3, with integers of coefficients.\\
If the scalars are any number in $\mathbb{R}$ then $c = \frac{1}{10}$, $p(t) = 1+x+x^2 \Rightarrow cp(t) = \frac{1}{10} + \frac{1}{10}x + \frac{1}{10}x^2 \notin$ the set.\\
But if we restrict scalars to the set of integers then it will be a subspace.\\
8. All polynomials in $\mathbb{P}_n$ such that $p(0) = 0$.\\
$\vec{0} = 0+0t+0t^2 \cdots , \vec{0} = 0$\\
For $u(t) = a_0 +a_1t+ \cdots , u(0) = 0$\\
$v(t) = b_0 + b_1t+ \cdots, v(0) = 0$\\
$u(t) + v(t) = a_0+b_0 + (a_1+b_1)t + \cdots , (u + v) = 0$\\
if $u(0) = 0$, $c \in \mathbb{R}$ then $cu(0) = c(0) = 0$\\
$\therefore$ is a subspace.\\
9. Let $H$ be the set of all vectors of the form $\begin{bmatrix}
-2t\\
5t\\
3t
\end{bmatrix}$. Find a vector $v$ in $\mathbb{R}^3$ such that $H = Span\{v\}$. Why does this show that $H$ is a subspace of $\mathbb{R}^3$?\\
$\vec{v} = t\begin{bmatrix}
-2\\
5\\
3
\end{bmatrix}$\\
$t = 0 \Rightarrow \begin{bmatrix}
-2(0)\\
5(0)\\
3(0)
\end{bmatrix} = \begin{bmatrix}
0\\
0\\
0
\end{bmatrix} \in H$, so $\vec{0} \in H$\\
$\vec{u} = \begin{bmatrix}
-2t_1\\
5t_1\\
3t_1
\end{bmatrix}$, $\vec{v} = \begin{bmatrix}
-2t_2\\
5t_2\\
3t_2
\end{bmatrix} \Rightarrow \vec{u} + \vec{v} = \begin{bmatrix}
-2(t_1+t_2)\\
5(t_1+t_2\\
3(t_1+t_2)
\end{bmatrix} \in H$\\
$\forall c \in \mathbb{R}$ $c \vec{u} = \begin{bmatrix}
-2(ct)\\
5(ct)\\
3(ct)
\end{bmatrix} \in H$\\
$\therefore H$ is a subspace.\\

Which of the following sets are linearly independent?\\
Which form a basis for $\mathbb{R}^3$\\
$\begin{bmatrix}
1\\
0\\
0
\end{bmatrix}$, $\begin{bmatrix}
2\\
3\\
0
\end{bmatrix}$
$\begin{bmatrix}
1&2&|&0\\
0&3&|&0\\
0&0&|&0
\end{bmatrix}$
$\frac{1}{2}R_2 = \begin{bmatrix}
1&2&|&0\\
0&1&|&0\\
0&0&|&0
\end{bmatrix}$
$R_1 - R_2 = \begin{bmatrix}
1&0&|&0\\
0&1&|&0\\
0&0&|&0
\end{bmatrix}$\\
$t$ has no free parameter.\\
$\therefore$ are linearly independent.\\
$\begin{bmatrix}
1\\
0\\
0
\end{bmatrix}$, $\begin{bmatrix}
2\\
3\\
0
\end{bmatrix}$, $\begin{bmatrix}
4\\
5\\
6
\end{bmatrix}$\\
$\begin{bmatrix}
1&2&4&|&0\\
0&3&5&|&0\\
0&0&6&|&0
\end{bmatrix}$
$\frac{1}{6}R_3 = \begin{bmatrix}
1&2&4&|&0\\
0&3&5&|&0\\
0&0&1&|&0
\end{bmatrix}$
$R_1 - 4R_3 = \begin{bmatrix}
1&2&0&|&0\\
0&3&5&|&0\\
0&0&1&|&0
\end{bmatrix}$
$R_2 - 5R_3 = \begin{bmatrix}
1&2&0&|&0\\
0&3&0&|&0\\
0&0&1&|&0
\end{bmatrix}$
$\frac{1}{3}R_2 = \begin{bmatrix}
1&2&0&|&0\\
0&1&0&|&0\\
0&0&1&|&0
\end{bmatrix}$
$R_1 - 2R_2 = \begin{bmatrix}
1&0&0&|&0\\
0&1&0&|&0\\
0&0&1&|&0
\end{bmatrix}$\\
$t$ has no free parameter, $\therefore$ are linearly independent.\\
$\begin{bmatrix}
1\\
0\\
0
\end{bmatrix}$, $\begin{bmatrix}
2\\
3\\
0
\end{bmatrix}$, $\begin{bmatrix}
4\\
5\\
6
\end{bmatrix}$, $\begin{bmatrix}
7\\
8\\
9
\end{bmatrix}$\\
4 Vectors in $\mathbb{R}^3$ must have at least 1 free parameter. $\therefore$ linearly dependent.\\
\end{document}